\documentclass[11pt]{article}
\usepackage{amsmath, amssymb}
\usepackage{hyperref}
\usepackage{graphicx}
\usepackage{multicol}

% Define \problemname and \illustration if not using Kattis template
\newcommand{\problemname}[1]{\section*{#1}}
\newcommand{\illustration}[3]{%
    \begin{figure}[h]
        \centering
        \includegraphics[width=#1\textwidth]{#2}
        \caption{#3}
    \end{figure}
}

\begin{document}

\problemname{Prefix Firewall}

% Example use of the \illustration command
% \illustration{0.3}{filename.jpg}{Photo by \href{url_here}{link text here, e.g. author name}}

You are a grey-hat hacker trying to identify security vulnerabilities in system firewalls. Each system you probe protects itself with a numeric firewall: a list of passcodes, one per line. A system is considered secure if every passcode in its firewall is polydivisible.
\\

\noindent A positive integer is polydivisible if each of its prefixes is divisible by its own length.
For example:
\begin{itemize}
    \item $381\,654\,729$ is polydivisible because each prefix length divides evenly:  
    $3 \bmod 1 = 0$, $38 \bmod 2 = 0$, $381 \bmod 3 = 0$, and so on.
    \item $26\,158$ is not polydivisible: although the first three prefixes ($2$, $26$, and $261$) are valid,  
    the fourth prefix $2615$ is not divisible by $4$.
    \item Any single-digit number is always polydivisible, since every integer is divisible by $1$.
\end{itemize}

\section*{Input}

Each input consists of one test case, representing a single secure system. The first line contains an integer $n$ ($1 \le n \le 1\,000$), the number of passcodes in the firewall. Each of the next $n$ lines contains one passcode, a positive integer consisting of between $1$ and $20$ digits (inclusive).

\section*{Output}

If every passcode line is polydivisible, output that the firewall is secure.  \\
If the firewall is not secure, print each non-polydivisible passcode in the order in which it appears.

\begin{multicols}{2}
\subsection*{Sample Input 1}
\begin{verbatim}
4
381654729
26
58
6
\end{verbatim}
\columnbreak
\subsection*{Sample Output 1}
\begin{verbatim}
not secure
26
58
\end{verbatim}
\end{multicols}

\begin{multicols}{2}
\subsection*{Sample Input 2}
\begin{verbatim}
3
381654729
6
8
\end{verbatim}
\columnbreak
\subsection*{Sample Output 2}
\begin{verbatim}
secure
\end{verbatim}
\end{multicols}

\begin{multicols}{2}
\subsection*{Sample Input 3}
\begin{verbatim}
2
10
12
\end{verbatim}
\columnbreak
\subsection*{Sample Output 3}
\begin{verbatim}
not secure
10
12
\end{verbatim}
\end{multicols}
\end{document}