\documentclass[11pt]{article}
\usepackage{amsmath, amssymb}
\usepackage{hyperref}
\usepackage{graphicx}
\usepackage{multicol}

% Define \problemname and \illustration if not using Kattis template
\newcommand{\problemname}[1]{\section*{#1}}
\newcommand{\illustration}[3]{%
    \begin{figure}[h]
        \centering
        \includegraphics[width=#1\textwidth]{#2}
        \caption{#3}
    \end{figure}
}

\begin{document}

\problemname{Prefix Firewall}

\illustration{0.3}{problem_statement/firewall_image.png}
% {Photo by \href{url_here}{link text here, e.g. author name}}

\noindent You are a grey-hat hacker trying to identify security vulnerabilities in system firewalls.
Each system you probe protects itself with a numeric firewall: a list of passcodes, one per line. The system is considered secure only if every passcode satisfies a special cyclic prefix-divisibility rule.
\\

\noindent Let $s = s_1 s_2 \dots s_d$ be the decimal digits of a positive integer.  
For each prefix length $k$ ($1 \le k \le d$), let $p_k$ be the integer formed by the first $k$ digits.  
The firewall uses a repeating modulus cycle to check divisibility:
\[ 1, 2, 3, \dots, 9, 10, 1, 2, \dots \]
so the modulus $m_k$ for prefix $p_k$ is such that $m_k = ((k-1) \bmod 10) + 1$ \\

\noindent A passcode is considered cyclic-polydivisible if and only if
\[ p_k \bmod m_k = 0 \quad \text{for all } k = 1,2,\dots,d \]

\noindent
For example:
\begin{itemize}
    \item $38\,165\,472$ is polydivisible because each prefix is divisible by the appropriate $m_k$: \\
    $3 \bmod 1 = 0$, $38 \bmod 2 = 0$, $381 \bmod 3 = 0$, and so on.
    \item $26\,158$ is not polydivisible: although the first three prefixes ($2$, $26$, and $261$) are valid,  the fourth prefix $2615$ is not divisible by $4$.
    \item Any single-digit number is always polydivisible, since every integer is divisible by $1$.
    \item A passcode fails immediately at the first prefix that is not divisible by its required modulus.
\end{itemize}

\section*{Input}

Each input consists of one test case, representing a single secure system. The first line contains an integer $n$ ($1 \le n \le 1\,000$), the number of passcodes in the firewall. Each of the next $n$ lines contains one passcode, a positive integer consisting of between $1$ and $5\,000$ digits (inclusive).

\section*{Output}

If every passcode line is polydivisible, output that the firewall is secure.  \\
If the firewall is not secure, print "not secure" followed by each non-polydivisible passcode in the order in which it appears.


\noindent \newline \newline
\begin{minipage}[t]{0.48\textwidth}
\subsection*{Sample Input 1}
\begin{verbatim}
4
38165472
4
666450405060
26
\end{verbatim}
\end{minipage}\hfill
\begin{minipage}[t]{0.48\textwidth}
\subsection*{Sample Output 1}
\begin{verbatim}
secure
\end{verbatim}
\end{minipage}

\noindent \newline \newline
\begin{minipage}[t]{0.48\textwidth}
\subsection*{Sample Input 2}
\begin{verbatim}
4
325
381654729
63
8
\end{verbatim}
\end{minipage}\hfill
\begin{minipage}[t]{0.48\textwidth}
\subsection*{Sample Output 2}
\begin{verbatim}
not secure
325
63
\end{verbatim}
\end{minipage}

\noindent \newline \newline
\begin{minipage}[t]{0.48\textwidth}
\subsection*{Sample Input 3}
\begin{verbatim}
3
1
38
123
\end{verbatim}
\end{minipage}\hfill
\begin{minipage}[t]{0.48\textwidth}
\subsection*{Sample Output 3}
\begin{verbatim}
secure
\end{verbatim}
\end{minipage}

\end{document}